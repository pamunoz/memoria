% Created 2016-11-10 jue 15:19
\documentclass[11pt]{article}
\usepackage[utf8]{inputenc}
\usepackage{lmodern}
\usepackage[T1]{fontenc}
\usepackage{fixltx2e}
\usepackage{graphicx}
\usepackage{longtable}
\usepackage{float}
\usepackage{wrapfig}
\usepackage{rotating}
\usepackage[normalem]{ulem}
\usepackage{amsmath}
\usepackage{textcomp}
\usepackage{marvosym}
\usepackage{wasysym}
\usepackage{amssymb}
\usepackage{amsmath}
\usepackage[version=3]{mhchem}
\usepackage[numbers,super,sort&compress]{natbib}
\usepackage{natmove}
\usepackage{url}
\usepackage{minted}
\usepackage{underscore}
\usepackage[linktocpage,pdfstartview=FitH,colorlinks,
linkcolor=blue,anchorcolor=blue,
citecolor=blue,filecolor=blue,menucolor=blue,urlcolor=blue]{hyperref}
\usepackage{attachfile}
\author{Pablo Farías}
\date{Jueves 10 de Octubre, 2016}
\title{Resumen Bibliográfico}
\begin{document}

\tableofcontents

\maketitle

\section{La adopción de la biotecnología moderna y su compatibilidad con una agricultura sustentable.}
\label{sec:orgheadline1}

De acuerdo a Ortíz \cite{ortiz-2012-la-de} la biotecnología es necesaria por una creciente demanda. Además es mucho mas practica para la medicina, la agricultura y la industria. Además la agrobiotecnología es más eficiente en el uso de los recursos y en la caracterización de las plagas y enfermedades de los cultivos.

Considera a la biotecnología como una evolución de los métodos tradicionales de cultivo, e incluso métodos biológicos parecidos ya se han utilizado con productos panaderos, cerveceros, vinicultores y de quesos.

La biotecnología es el aislamiento del ADN en un laboratorio para luego insertarlo en otro organismo. Para Ortíz la biotecnología debe ser parte del mejoramiento tecnológico moderno en la industria alimenticia.

Las biotecnología ofrece semillas resistentes a plagas y enfermedades, con niveles reducidos de micotoxinas y con resistencia a herbicidas siendo mas útiles a la agricultura de conservación. 

El "arroz dorado" por ejemplo contiene B-caroteno, que sirve para la mortalidad infantil y la ceguera asociada a la deficiencia de vitamina A(VAD).

Los cultivos transgénicos también son útiles para la mitigación del cambio climático. Más de la mitad de los fertilizantes se volatiliza como óxido nitroso ( aproximadamente el 6\% de las emisiones de gases invernaderos) además de contaminar las aguas y causar zonas muertas.

Un importante uso que se les ha dado a los cultivos transgénicos es el control de plagas a través  de la inserción del gen productor de la toxina \emph{Bacillus thuringiensis (Bt)}.  Esa toxina también controla la propagación de las plagas a los cultivos vecinos. 

También la resistencia ha herbicidas por la introducción de ciertos genes como el \emph{glifosato}.

Es decir, el objetivo actual de los transgénicos es la salud de los cultivos por medio de la resistencia a las plagas, virus y mayor tolerancia a los herbicidas.

El 2011 se sembraron 160 millones de hectáreas de cultivos transgénicos en 29 países del mundo. Con esto aporta al combate del cambio climático debido a que la disminución de los herbicidas que producen gases de efecto invernadero. También su eficiencia a permitido alimentar a mayor población usando menos hectáreas.

En la India el algodón- \emph{Bt} ha aumentado su rendimiento y con esto el 50\% de las ganancias de los pequeños agricultores de ese país. 

El debate debe centrarse en como regular esta tecnología para que no tenga impactos negativos sociales y en la biodiversidad tanto animal como alimenticia. Además se deben considerar la gran variedad de productos ya utilizados actual mente por agricultores y el resto de la población.

Los principales alimentos utilizados actualmente ya han sido considerados seguros para su consumo (principalmente maíz, soja y colza) y los métodos para este resultado han sido aprobados por el Consejo Internacional de la Ciencia (ICSU) y en común acuerdo con la Organización Mundial de la Salud (OMS) junto con la Organización de las Naciones Unidas para la Agricultura y la Alimentación (FAO).

Los OGM han pasado todos las pruebas sobre la salud humana y es improbable que presenten efectos negativos inesperados. Y en los países que en los que ya hace década y media ya han sido introducidos no han presentado efectos sobre la salud humana, incluyendo los riesgos potenciales.





\bibliographystyle{unsrt}
\bibliography{bibliografia}
\end{document}